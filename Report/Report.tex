\documentclass[11pt,a4paper]{scrartcl}
\usepackage[utf8]{inputenc}
\usepackage[english]{babel}
\usepackage{microtype}
\usepackage{amsmath}
\usepackage{mathtools}
%\usepackage[leqno]{amsmath}
\usepackage{amsfonts}
\usepackage{amssymb}
\usepackage{graphicx}
\usepackage{indentfirst}
\usepackage{fancyvrb}
%\usepackage{subfigure}
\usepackage{caption}
\usepackage{subcaption}
\usepackage{enumitem}
\usepackage{booktabs}
\usepackage{algorithm}
\usepackage{algpseudocode}
\usepackage[onehalfspacing]{setspace}
\usepackage[hidelinks]{hyperref}
\usepackage{listings}
\usepackage{xcolor}
\usepackage{fancyhdr}
%\usepackage{tikz}
%\usetikzlibrary{shapes,arrows}
\pagestyle{fancy}
\fancyhf{}
\fancyhead[R]{\thepage}

% Giuseppe L'Erario

\author{Giuseppe L'Erario}
\date{}
\title{DDPG for cheetah}
\subject{Reinforcement learning}
\begin{document}

\maketitle
\tableofcontents

\clearpage

\section{Introduction}

Reinforcement learning is a group of algorithm that take as inspiration the evolution and the learning in the natural world. It means that it is based on a trial and error logic. In particular the goal is to build an agent that learns how to act in a specific environment on the base of the reward it receives during the process.

\textbf{Some stories on the algorithms}...

These algorithms have as output a set of discrete actions. It is simple to imagine that, as the output space grows, the problem became intractable. In particular we have to associate to a state thousands of possible discrete actions. And, as one can expect, the possible state are thousands too!

A solution to this problem (course of dimensionality?) is developed by deep mind in the paper (cite paper). 

The paper takes the origin from the Deterministic Policy gradient algorithm, improved with techniques coming from the deep learning. 

\section{The state of the art}

\section{The algorithm}
The Deep Deterministic Policy Gradient algorithm uses an actor-critic approach. It is an off policy rl algorithm. It uses the Bellman equation to learn the Q-function and this Q-function to learn the policy.




\section{Results}

\section{Conclusions}

\end{document}
